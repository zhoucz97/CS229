
\newcommand{\argminE}{\mathop{\mathrm{argmin}}\limits}          % ASdeL
\newcommand{\argmaxE}{\mathop{\mathrm{argmax}}\limits}          % ASdeL

\clearpage

\item \points{25} {\bf KL Divergence, Fisher Information, and the Natural Gradient}

As seen before, the Kullback-Leibler divergence between two distributions is an asymmetric measure of how different two distributions are. Consider two distributions over the same space given by densities $p(x)$ and $q(x)$. The KL divergence between two continuous distributions, $q$ and $p$ is defined as,
 \begin{align*}
\KL(p||q)&=\int_{-\infty}^{\infty}p(x)\log\dfrac{p(x)}{q(x)}dx\\
&=\int_{-\infty}^{\infty}p(x)\log p(x)dx-\int_{-\infty}^{\infty}p(x)\log q(x)dx\\
&=\mathbb{E}_{x\sim p(x)}[\log p(x)]-\mathbb{E}_{x\sim p(x)}[\log q(x)].
\end{align*}


A nice property of KL divergence is that it invariant to parametrization. This means, KL divergence
evaluates to the same value no matter how we parametrize the distributions $P$ and $Q$. For e.g,
if $P$ and $Q$ are in the exponential family, the KL divergence between them is the same whether
we are using natural parameters, or canonical parameters, or any arbitrary reparametrization.


Now we consider the problem of fitting model parameters using gradient descent (or stochastic gradient
descent). As seen previously, fitting model parameters using Maximum Likelihood is equivalent
to minimizing the KL divergence between the data and the model. While KL divergence is
invariant to parametrization, the gradient w.r.t the model parameters (i.e, direction
of steepest descent) is \emph{not invariant to parametrization}. To see its implication, suppose
we are at a particular value of parameters (either randomly initialized, or mid-way through
the optimization process). The value of the parameters correspond to some probability distribution
(and in case of regression, a conditional probability distribution).
If we follow the direction of steepest descent from the current parameter, take a small step along that
direction to a new parameter, we end up with a new distribution corresponding to the new parameters.
The non-invariance to reparametrization means, a step of fixed size in the parameter space could
end up in a distribution that could either be extremely far away in $\KL$ from the previous
distribution, or on the other hand not move very much at all w.r.t $\KL$ from the previous
distributions.


This is where the \emph{natural gradient} comes into picture. It is best introduced in contrast
with the usual gradient descent. In the usual gradient descent, we \emph{first choose the direction}
by calculating the gradient of the MLE objective w.r.t the parameters, and then move a magnitude of
step size (where size is measured in the \emph{parameter space}) along that direction. Whereas
in natural gradient, we \emph{first choose a divergence} amount by which we would like to
move, in the $\KL$ sense. This effectively gives us a perimeter around the current parameters (of
some arbitrary shape), such that points along this perimeter correspond to distributions
which are at an equal $\KL$-distance away from the current parameter. Among the set
of all distributions along this perimeter, we move to the distribution that maximizes the
objective (i.e minimize $\KL$ between data and itself) the most. This approach makes the optimization
process invariant to parametrization. That means, even if we chose a new arbitrary reparametrization,
by starting from a particular distribution, we always descend down the same sequence of
distributions towards the optimum.


In the rest of this problem, we will construct and derive the natural gradient update rule.
For that, we will break down the process into smaller sub-problems, and give you hints
to answer them. Along the way, we will encounter important statistical concepts such
as the \emph{score function} and \emph{Fisher Information} (which play a prominant role
in Statistical Learning Theory as well). Finally, we will see how this new natural gradient
based optimization is actually equivalent to Newton's method for Generalized Linear Models.


Let the distribution of a random variable $Y$ parameterized by $\theta \in \mathbb{R}^{n}$ be $p(y;\theta)$.

\begin{enumerate}
\item \subquestionpoints{3} \textbf{Score function}

The score function associated with $p(y;\theta)$ is defined as $\nabla_{\theta}\log p(y;\theta)$, which signifies the sensitivity of the likelihood function with respect to the parameters. Note that the score function is actually a vector since it's the gradient of a scalar quantity with respect to the vector $\theta$. 

Recall that $\mathbb{E}_{y\sim p(y)}[g(y)]=\int_{-\infty}^{\infty}p(y)g(y)dy$. Using this fact, show that the expected value of the score is 0, i.e. 

$$\mathbb{E}_{y\sim p(y;\theta)}[\nabla_{\theta'} \log p(y;\theta')|_{\theta'=\theta}]=0$$


\ifnum\solutions=1 {
  \input{03-natural_grad/01-exp_score_sol}
} \fi


\ifnum\solutions=1 {
  \clearpage
} \fi
\item \subquestionpoints{2} \textbf{Fisher Information}

Let us now introduce a quantity known as the Fisher information. It is defined as the covariance matrix of the score function,
$$\mathcal{I}(\theta) = \text{Cov}_{y \sim p(y;\theta)}[\nabla_{\theta'}\log p(y;\theta')|_{\theta'=\theta}]$$

Intuitively, the Fisher information represents the amount of information that a random variable $Y$ carries about a parameter $\theta$ of interest. When the parameter of interest is a vector (as in our case, since $\theta \in \mathbb{R}^n$), this information becomes a matrix. Show that the Fisher information can equivalently be given by

$$\mathcal{I}(\theta)=\mathbb{E}_{y\sim p(y;\theta)}[\nabla_{\theta'} \log p(y;\theta')\nabla_{\theta'} \log p(y;\theta')^\top|_{\theta'=\theta}]$$

Note that the Fisher Information is a function of the parameter. The parameter of the Fisher information is both a) the parameter value at which the score function is evaluated, and b) the parameter of the distribution with respect to which the expectation and variance is calculated.

\ifnum\solutions=1 {
  \input{03-natural_grad/02-cov_score_sol}
} \fi


\ifnum\solutions=1 {
  \clearpage
} \fi
\item \subquestionpoints{5} \textbf{Fisher Information (alternate form)}

It turns out that the Fisher Information can not only be defined as the covariance of the score function,
but in most situations it can also be represented as the expected negative Hessian of the log-likelihood.

Show that $\mathbb{E}_{y\sim p(y;\theta)}[-\nabla^2_{\theta'} \log p(y;\theta')|_{\theta'=\theta}]=\mathcal{I}(\theta)$.

\ifnum\solutions=1 {
  \input{03-natural_grad/03-nhess_score_sol}
} \fi

\textbf{Remark}. The Hessian represents the curvature of a function at a point. This shows that the expected curvature of the log-likelihood function is also equal to the Fisher information matrix. If the curvature of the log-likelihood at a parameter is very steep (i.e, Fisher Information is very high), this generally means you need fewer number of data samples to a estimate that parameter well (assuming data was generated from the distribution with those parameters), and vice versa. The Fisher information matrix associated with a statistical model parameterized by $\theta$ is extremely important in determining how a model behaves as a function of the number of training set examples.


\ifnum\solutions=1 {
  \clearpage
} \fi
\item \subquestionpoints{5} \textbf{Approximating $\KL$ with Fisher Information}

As we explained at the start of this problem, we are interested in the set of all distributions that are at a small fixed $\KL$ distance away from the current distribution. In order to calculate $\KL$ between $p(y;\theta)$ and $p(y;\theta+d)$, where $d \in \mathbb{R}^n$ is a small magnitude ``delta'' vector, we approximate it using the Fisher Information at $\theta$. Eventually $d$ will be the natural gradient update we will add to $\theta$. To approximate the KL-divergence with Fisher Infomration, we will start with the Taylor Series expansion of $\KL$ and see that the Fisher Information pops up in the expansion.

Show that $\KL(p_{\theta}||p_{\theta+d})\approx \dfrac{1}{2}d^T\mathcal{I}(\theta)d$.

Hint: Start with the Taylor Series expansion of $\KL(p_{\theta}||p_{\tilde{\theta}})$ where $\theta$ is a constant and $\tilde{\theta}$ is a variable. Later set $\tilde{\theta}= \theta + d$. Recall that the Taylor Series allows us to approximate a scalar function $f(\tilde{\theta})$ near $\theta$ by:
\begin{align*}
    f(\tilde{\theta})\approx f(\theta)+(\tilde{\theta}-\theta)^T\nabla_{\theta'} f(\theta')|_{\theta'=\theta} + \frac{1}{2}(\tilde{\theta}-\theta)^T \left(\nabla^2_{\theta'}f(\theta')|_{\theta'=\theta}\right) (\tilde{\theta}-\theta)
\end{align*}

\ifnum\solutions=1 {
  \input{03-natural_grad/04-kl_taylor_sol}
} \fi


\ifnum\solutions=1 {
  \clearpage
} \fi
\item \subquestionpoints{8} \textbf{Natural Gradient}

Now we move on to calculating the natural gradient. Recall that we want to maximize the log-likelihood by moving only by a fixed $\KL$ distance from the current position. In the previous sub-question we came up with a way to approximate $\KL$ distance with Fisher Information. Now we will set up the constrained optimization problem that will yield the natural gradient update $d$. Let the log-likelihood objective be $\ell(\theta) = \log p(y;\theta)$. Let the $\KL$ distance we want to move by, be some small positive constant $c$. The natural gradient update $d^*$ is

\begin{align*}
d^* &= \arg\max_d \ell(\theta + d) \quad \text{subject to} \quad \KL(p_\theta||p_{\theta + d}) = c & \text{(1)}
\end{align*}

First we note that we can use Taylor approximation on $\ell(\theta + d) \approx \ell(\theta) + d^T\nabla_{\theta'}\ell(\theta')|_{\theta'=\theta}$. Also note that we calculated the Taylor approximation $\KL(p_\theta||p_{\theta+d})$ in the previous subproblem. We shall substitute both these approximations into the above constrainted optimization problem.

In order to solve this constrained optimization problem, we employ the \emph{method of Lagrange multipliers}. If you are familiar with Lagrange multipliers, you can proceed directly to solve for $d^*$. If you are not familiar with Lagrange multipliers, here is a simplified introduction. (You may also refer to a slightly more comprehensive introduction in the Convex Optimization section notes, but for the purposes of this problem, the simplified introduction provided here should suffice).

Consider the following constrained optimization problem
$$d^\ast =\arg\max_d f(d) \quad \text{subject to} \quad g(d)=c$$
The function $f$ is the objective function and $g$ is the constraint. We instead optimize the \emph{Lagrangian} $\mathcal{L}(d,\lambda)$, which is defined as

$$\mathcal{L}(d,\lambda) = f(d) - \lambda [ g(d)-c ]$$

with respect to both $d$ and $\lambda$. Here $\lambda \in \mathbb{R}_+$ is called the Lagrange multiplier. In order to optimize the above, we construct the following system of equations:

\begin{align*}
 \nabla_d \mathcal{L}(d,\lambda) &= 0, &\text{(a)} \\
 \nabla_\lambda \mathcal{L}(d,\lambda) &= 0. &\text{(b)}
\end{align*}

So we have two equations (a and b above) with two unknowns ($d$ and $\lambda$), which can be sometimes be solved analytically (in our case, we can).

The following steps guide you through solving the constrained optimization problem:

\begin{itemize}
\item 
Construct the Lagrangian for the constrained optimization problem (1) with the Taylor approximations substituted in for both the objective and the constraint.



\item 
Then construct the system of linear equations (like (a) and (b)) from the Lagrangian you obtained.


\item

From (a), come up with an expression for $d$ that \emph{involves} $\lambda$.


At this stage we have already found the ``direction'' of the natural gradient $d$, since $\lambda$ is only a positive scaling constant. For most practical purposes, the solution we obtain here is sufficient. This is because we almost always include a learning rate hyperparameter in our optimization algorithms, or perform some kind of a line search for algorithmic stability. This can make the exact calculation of $\lambda$ less critical. Let's call this expression $\tilde{d}$ (involving $\lambda$) as the \emph{unscaled natural gradient}. Clearly state what is $\tilde{d}$ as a function of $\lambda$.


The remaining steps are to figure out the value of the scaling constant $\lambda$ along the direction of $d$, for completeness.

\item 

Plugin that expression for $d$ into (b). Now we have an equation that has $\lambda$ but not $d$. Come up with an expression for $\lambda$ that does \emph{not include} $d$.

\item

Plug that expression for $\lambda$ (without $d$) back into (a). Now we have an equation that has $d$ but not $\lambda$. Come up with an expression for $d$ that does \emph{not include} $\lambda$.

\end{itemize}


The expression fof $d$ obtained this way will be the desired natural gradient update $d^*$. Clearly state and highlight your final expression for $d^*$. This expression cannot include $\lambda$.

\ifnum\solutions=1 {
  \input{03-natural_grad/05-lagrange_sol}
} \fi


\ifnum\solutions=1 {
  \clearpage
} \fi
\item \subquestionpoints{2} \textbf{Relation to Newton's Method}


After going through all these steps to calculate the natural gradient, you might wonder if this is something used in practice. We will now see that the familiar Newton's method that we studied earlier, when applied to Generalized Linear Models, is equivalent to natural gradient on Generalized Linear Models. While the two methods (Netwon's and natural gradient) agree on GLMs, in general they need not be equivalent.


Show that the direction of update of Newton's method, and the direction of natural gradient, are exactly the same for Generalized Linear Models. You may want to recall and cite the results you derived in problem set 1 question 4 (Convexity of GLMs). For the natural gradient, it is sufficient to use $\tilde{d}$, the unscaled natural gradient.

\ifnum\solutions=1 {
  \input{03-natural_grad/06-update_rule_sol}
} \fi


\end{enumerate}
