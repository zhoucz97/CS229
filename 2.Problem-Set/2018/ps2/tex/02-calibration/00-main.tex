\clearpage
\item \points{10} {\bf Model Calibration}

In this question we will try to understand the output $h_\theta(x)$ of the
hypothesis function of a logistic regression model, in particular why we might
treat the output as a probability (besides the fact that the sigmoid function
ensures $h_\theta(x)$ always lies in the interval $(0, 1)$).

When the probabilities outputted by a model match empirical observation, the
model is said to be \emph{well calibrated} (or reliable). For example, if we
consider a set of examples $x^{(i)}$ for which $h_\theta(x^{(i)})  \approx
0.7$, around 70\% of those examples should have positive labels. In a
well calibrated model, this property will hold true at every probability value.

Logistic regression tends to output well calibrated probabilities (this is
often not true with other classifiers such as Naive Bayes, or SVMs). We will
dig a little deeper in order to understand why this is the case, and find that
the structure of the loss function explains this property.

Suppose we have a training set $\{x^{(i)},y^{(i)}\}_{i=1}^m$ with
$x^{(i)} \in \mathbb{R}^{n+1}$ and $y^{(i)} \in \{0, 1\}$. Assume we have an
intercept term $x_0^{(i)} = 1$ for all $i$. Let $\theta \in \mathbb{R}^{n+1}$
be the maximum likelihood parameters learned after training a logistic
regression model. In order for the model to be considered well calibrated,
given any range of probabilities $(a, b)$ such that $0 \le a < b \le 1$, and
training examples $x^{(i)}$ where the model outputs $h_\theta(x^{(i)})$ fall in
the range $(a, b)$, the fraction of positives in that set of examples should be
equal to the average of the model outputs for those examples. That is, the
following property must hold:
$$
  \frac{\sum_{i\in I_{a,b}}  P\left(y^{(i)}=1|x^{(i)};\theta\right)}
       {{|\{i\in I_{a,b}\}|}}
  = \frac{\sum_{i\in I_{a,b}} \mathbb{I}\{y^{(i)} = 1\}}
         {|\{i\in I_{a,b}\}|},
$$
where $P(y=1|x;\theta) = h_\theta(x) = 1/(1+\exp(-\theta^\top x))$, $I_{a,b} =
\{ i | i \in \{1,...,m\},  h_\theta(x^{(i)}) \in (a,b)\} $ is an index set of
all training examples $x^{(i)}$ where $h_\theta(x^{(i)}) \in (a,b)$, and $|S|$
denotes the size of the set $S$.

\begin{enumerate}
  \item \subquestionpoints{10}
In lecture we saw the average empirical loss for logistic regression:
\begin{equation*}
	J(\theta)
	= -\frac{1}{m} \sum_{i=1}^m y^{(i)}\log(h_{\theta}(x^{(i)}))
		+  (1 - y^{(i)})\log(1 - h_{\theta}(x^{(i)})),
\end{equation*}
where $y^{(i)} \in \{0, 1\}$, $h_\theta(x) = g(\theta^T x)$ and
$g(z) = 1 / (1 + e^{-z})$.

Find the Hessian $H$ of this function, and show that for any vector $z$, it
holds true that
%
\begin{equation*}
    z^T H z \ge 0.
\end{equation*}
%
{\bf Hint:} You may want to start by showing that
$\sum_i\sum_j z_i x_i x_j z_j = (x^Tz)^2 \geq 0$. Recall also that
$g'(z) = g(z)(1-g(z))$.

{\bf Remark:} This is one of the standard ways of showing that the matrix $H$
is positive semi-definite, written ``$H \succeq 0$.''  This implies that $J$ is
convex, and has no local minima other than the global one. If you have some
other way of showing $H \succeq 0$, you're also welcome to use your method
instead of the one above.

\ifnum\solutions=1 {
  \input{01-linreg/01-logreg-sol}
} \fi

  \item \subquestionpoints{3}
If we have a binary classification model that is perfectly calibrated---that
is, the property we just proved holds for any $(a, b) \subset [0, 1]$---does
this necessarily imply that the model achieves perfect accuracy? Is the
converse necessarily true? Justify your answers.

\ifnum\solutions=1 {
  \input{02-calibration/02-accuracy-sol}
} \fi

  \item \subquestionpoints{2}
Discuss what effect including $L_2$ regularization in the logistic regression
objective has on model calibration.

\ifnum\solutions=1 {
  \input{02-calibration/03-l2-effect-sol}
} \fi

\end{enumerate}

\textbf{Remark:}  We considered the range $(a,b) = (0, 1)$. This is the only
range for which logistic regression is guaranteed to be calibrated on the
training set. When the GLM modeling assumptions hold, all ranges $(a,b) \subset
[0,1]$ are well calibrated. In addition, when the training and test set are
from the same distribution and when the model has not overfit or underfit,
logistic regression tends to be well calibrated on unseen test data as well.
This makes logistic regression a very popular model in practice, especially
when we are interested in the level of uncertainty in the model output.
